\chapter{Groups}


\section{Semigroups, Monoids and Groups}

\subsection{Exercises}

\subsubsection*{1}
\begin{graybox}
	Give examples other than those in the text of semigroups and monoids that are not groups
\end{graybox}
\begin{solution}
	content...
\end{solution}

\subsubsection*{2}
\begin{graybox}
	Let $G$ be a group (written additively), $S$ a nonempty set, and $M(S,G)$ the set of all functions $f : S \to G$. Define addition in $M(S,G)$ as follows: $(f + g) : S \to G$ is given by $s \mapsto f(s) + g(s) \in G$. Prove that $M(S,G)$ is a group, which is abelian if $G$ is.
\end{graybox}
\begin{solution}
	content...
\end{solution}

\subsubsection*{3}
\begin{graybox}
	Is it true that a semigroup which has a \textit{left} identity element and in which every element has a \textit{right} inverse is a group?
\end{graybox}
\begin{solution}
	content...
\end{solution}

\subsubsection*{4}
\begin{graybox}
	Write out a multiplication table for the group $D_4^*$
\end{graybox}
\begin{solution}
	content...
\end{solution}

\subsubsection*{5}
\begin{graybox}
	Prove that the symmetric group on $n$ letters, $S_n$ has order $n!$.
\end{graybox}
\begin{solution}
	content...
\end{solution}

\subsubsection*{6}
\begin{graybox}
	Write out an addition table for $\ZZ_2 \oplus \ZZ_2$, called the \textbf{Klein four group.}
\end{graybox}
\begin{solution}
	content...
\end{solution}

\subsubsection*{7}
\begin{graybox}
	If $p$ is prime, then the nonzero elements of $\ZZ_p$ form a group of order $p - 1$ under multiplication. Show that this statement is false if $p$ is not prime.
\end{graybox}
\begin{solution}
	content...
\end{solution}

\subsubsection*{8}
\begin{enumerate}[(a)]
	\item 
	\begin{graybox}
		The relation given by $a \sim a \iff a - b \in \ZZ$ is a congruence relation on the additive group $\QQ$
	\end{graybox}
	\begin{solution}
		content...
	\end{solution}
	
	\item 
	\begin{graybox}
		The set $\QQ/\ZZ$ of equivalence classes is an infinite abelian group.
	\end{graybox}
	\begin{solution}
		content...
	\end{solution}
\end{enumerate}

\subsubsection*{9}
\begin{graybox}
	Let $p$ be a fixed prime. Let $R_p$ be the set of all those rational numbers whose denominator is relatively prime to $p$. Let $R^p$ be the set of rationals whose denominator is a power of $p$ ($p^i, i > 0$). Prove that both $R_p$ and $R^p$ are abelian groups under ordinary addition of rationals. 
\end{graybox}
\begin{solution}
	content...
\end{solution}

\subsubsection*{10}
\begin{graybox}
	Let $p$ be a prime and let $Z(p^\infty)$ be the following subset of the group $\QQ/\ZZ$:
	$$
		Z(p^\infty) = \{\overline{a/b} \in \QQ/\ZZ \mid a,b \in \ZZ \text{ and } b = p^i \text{ for some } i \geq 0\}.
	$$
	Show that $Z(p^\infty)$ is an infinite group under the addition operation of $\QQ/\ZZ$.
\end{graybox}
\begin{solution}
	content...
\end{solution}

\subsubsection*{11}
\begin{graybox}
	The following conditions on a group $G$ are equivalent
	\begin{enumerate}[(i)]
		\item $G$ is abelian
		\item $(ab)^2 = a^2b^2$ for all $a,b \in G$
		\item $(ab)^{-1} = a^{-1}b^{-1}$ for all $a,b \in G$
		\item $(ab)^n = a^nb^n$ for all $n \in \ZZ$ and all $a, b \in G$
		\item $(ab)^n = a^nb^n$ for three consecutive integers $n$ and all $a,b \in G$.
	\end{enumerate}
\end{graybox}
\begin{solution}
	content...
\end{solution}
\begin{lightgraybox}
	Show that (v)$\implies(i)$ is false if ``three'' is replaced by ``two''
\end{lightgraybox}
\begin{solution}
	content...
\end{solution}

\subsubsection*{12}
\begin{graybox}
	If $G$ is a group, $a, b\in G$ and $bab^{-1} = r$ for some $r \in \NN$, then $b^jab^{-j} = a^{r^j}$ for all $j \in \NN$.
\end{graybox}

\subsubsection*{13}
\begin{graybox}
	If $a^2 = e$ for all elements $a$ of a group $G$, then $G$ is abelian.
\end{graybox}
\begin{solution}
	content...
\end{solution}

\subsubsection*{14}
\begin{graybox}
	If $G$ is a finite group of even order, then $G$ contains an element $a \neq e$ such that $a^2 = e$
\end{graybox}
\begin{solution}
	content...
\end{solution}

\subsubsection*{15}
\begin{graybox}
	Let $G$ be a nonempty finite set with an associative binary operation such that for all $a,b,c \in G, ab = ac \implies b = c$ and $ba = ca \implies b = c$. Then $G$ is a group. 
\end{graybox}
\begin{solution}
	content...
\end{solution}
\begin{lightgraybox}
	Show that this conclusion may be false if $G$ is infinite.
\end{lightgraybox}
\begin{solution}
	content...
\end{solution}

\subsubsection*{16}
\begin{graybox}
	Let $a_1, a_2,\dots$ be a sequence of elements in a semigroup $G$. THen there exists a unique function $\psi : \NN^* \to G$ such that $\psi(1) = a_1, \psi(2) = a_1a_2, \psi(3) = (a_1a_2)a_3$ and for $n\geq 1, \psi(n + 1) = (\psi(n))a_{n+1}$. Note that $\psi(n)$ is precisely the standard $n$ product $\prod_{i=1}^na_i$
\end{graybox}
\begin{solution}
	content...
\end{solution}