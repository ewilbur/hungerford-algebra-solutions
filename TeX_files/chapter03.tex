\chapter{Rings}


\section{Rings and Homomorphisms}

\subsection{Exercises}

\subsubsection*{1}
\begin{enumerate}[(a)]
	\item
	\begin{graybox}
		Let $G$ be an (additive) abelian group. Define an operation of multiplication in $G$ by $ab = 0$ (for all $a, b\in G$). Then $G$ is a ring.
	\end{graybox}
	\begin{solution}
		We show $G$ is a ring directly from Definition 1.1
		\begin{enumerate}[(i)]
			\item Given
			
			\item Let $a,b,c \in G$
			\begin{align*}
				(ab)c &= 0c\\
				&= 0 & \text{Theorem 1.2 (i)}\\
				&= a0\\
				&= a(bc).
			\end{align*}
			Hence $G$ is associative.
			
			\item Let $a,b,c \in G$
			\begin{align*}
				a(b + c) &= 0\\
				&= 0 + 0\\
				&= ab + ac.
			\end{align*}
			Hence $G$ is distributive.
		\end{enumerate}
	\end{solution}
	\item
	\begin{graybox}
		Let $S$ be the set of all subsets of some fixed set $U$. For $A, B \in S$, define $A + B = (A \setminus B) \cup (B \setminus A)$ and $AB = A \cap B$. Then $S$ is a ring
	\end{graybox}
	\begin{solution}
		content...
	\end{solution}
	\begin{lightgraybox}
		Is $S$ commutative?
	\end{lightgraybox}
	
	\begin{solution}
		content...
	\end{solution}
	\begin{lightgraybox}
		Does it have an identity?
	\end{lightgraybox}
	\begin{solution}
		content...
	\end{solution}
\end{enumerate}

\subsubsection*{2}
\begin{graybox}
	Let $\{R_i \mid i \in I\}$ be a family of rings with identity. Make the direct sum of abelian groups $\sum_{i \in I} R_i$ into a ring by defining multiplication coordinatewise. Does $\sum_{i \in I}R_i$ have identity?
\end{graybox}
\begin{solution}
	content...
\end{solution}

\subsubsection*{3}
\begin{graybox}
	A ring $R$ such that $a^2 = a$ for all $a \in R$ is called a \textbf{Boolean ring}. Prove that every Boolean ring $R$ is commutative and $a + a = 0$ for all $a \in R$.
\end{graybox}
\begin{solution}
	For any $a, b \in R$,
	\begin{align*}
		a + b &= (a + b)^2\\
		&= (a + b)(a + b)\\
		&= a(a + b) + b(a + b)\\
		&= a^2 + ab + ba + b^2\\
		&= a + ab + ba + b;
	\end{align*}
	hence we have
	\begin{align*}
		0 &= ab + ba\\
		-ba &= ab\\
		(-ba)^2 &= ab\\
		(ba)^2 &= ab & \text{Theorem 1.2 (iii)}\\
		ba &= ab.
	\end{align*}
	Therefore $R$ is commutative.\\
	\\
	Additionally, for any $x \in R$
	\begin{align*}
		x + x &= (x + x)^2\\
		&= (x + x)(x + x)\\
		&= x(x + x) + x(x + x)\\
		&= x^2 + x^2 + x^2 + x^2\\
		&= x + x + x + x.
	\end{align*}
	Thus,
	$$
		0 = x + x.
	$$
\end{solution}

\subsubsection*{4}
\begin{graybox}
	Let $R$ be a ring and $S$ a nonempty set. Then the group $M(S,R)$ (Exercise I.1.2) is a ring with multiplication defined as follows: the product of $f, g \in M(S,R)$ is the function $S \to R$ given by $s \mapsto f(s)g(s)$.
\end{graybox}
\begin{solution}
	content...
\end{solution}

\subsubsection*{5}
\begin{graybox}
	If $A$ is the abelian group $\ZZ \oplus \ZZ$, then $\text{End} A$ is a noncommutative ring.
\end{graybox}
\begin{solution}
	content...
\end{solution}

\subsubsection*{6}
\begin{graybox}
	A finite ring with more than one element and no zero divisors is a division ring. (Special case: a finite integral domain is a field.)
\end{graybox}
\begin{solution}
	content...
\end{solution}


\subsubsection*{7}
\begin{graybox}
	Let $R$ be a ring with more than one element such that for each nonzero $a \in R$ there is a unique $b \in R$ such that $aba = a$. Prove:
\end{graybox}
\begin{enumerate}[(a)]
	\item 
	\begin{lightgraybox}
		$R$ has no zero divisors.
	\end{lightgraybox}
	\begin{solution}
		content...
	\end{solution}
	
	\item
	\begin{lightgraybox}
		$bab = b$.
	\end{lightgraybox}
	\begin{solution}
		content...
	\end{solution}
	
	\item
	\begin{lightgraybox}
		$R$ has an identity.
	\end{lightgraybox}
	\begin{solution}
		content...
	\end{solution}
	
	\item
	\begin{lightgraybox}
		$R$ is a division ring.
	\end{lightgraybox}
	\begin{solution}
		content...
	\end{solution}
\end{enumerate}

\subsubsection*{8}
\begin{graybox}
	Let $R$ be the set of all $2 \times 2$ matrices over the complex field $\CC$ of the form
	\[
	\begin{pmatrix}
		z & w \\
		-\overline{w} & \overline{z}
	\end{pmatrix},
	\]
	where $\overline{z}, \overline{w}$ are the complex conjugates of $z$ and $w$ respectively (that is, $c = a + b\sqrt{-1} \iff \overline{c} = a - b\sqrt{-1}$). Then $R$ is a division ring that is isomorphic to the division ring $K$ of real quaternions. \textit{[Hint: Define an isomorphism $K \to R$ by letting the images of $1, i, j, k \in K$ be respectively the matrices}
	\[
	\begin{pmatrix}
		1 & 0 \\
		0 & 1
	\end{pmatrix}, \quad
	\begin{pmatrix}
		0 & \sqrt{-1} \\
		\sqrt{-1} & 0
	\end{pmatrix}, \quad
	\begin{pmatrix}
		0 & 1 \\
		-1 & 0
	\end{pmatrix}, \quad
	\begin{pmatrix}
		0 & \sqrt{-1} \\
		-\sqrt{-1} & 0
	\end{pmatrix}.
	\]
\end{graybox}
\begin{solution}
	content...
\end{solution}

\subsubsection*{9}
\begin{enumerate}[(a)]
	\item 
	\begin{graybox}
		The subset $G = \{1,-1,i,-i,j,-j,k,-k\}$ of the division ring $K$ of real quaternions forms a group under multiplication.
	\end{graybox}
	\begin{solution}
		content...
	\end{solution}
	
	\item
	\begin{graybox}
		$G$ is isomorphic to the quaternion group (Exercises I.4.14 and I.2.3)
	\end{graybox}
	\begin{solution}
		content...
	\end{solution}
	
	\item
	\begin{graybox}
		What is the difference between the ring $K$ and the group ring $\RR(G)$ ($\RR$ the field of real numbers?)
	\end{graybox}
	\begin{solution}
		content...
	\end{solution}
\end{enumerate}

\subsubsection*{10}
\begin{graybox}
	Let $k,n \in \ZZ$ such that $0 \leq k \leq n$ and ${n \choose k}$ the binomial coefficient
	$$
		\frac{n!}{(n - k)!k!},
	$$
	where $0! = 1$ and for $n > 0, n! = \prod_{i=1}^{n} i$.
\end{graybox}
\begin{enumerate}[(a)]
	\item 
	\begin{lightgraybox}
		${n \choose k} = {n \choose n-k}$
	\end{lightgraybox}
	\begin{solution}
		content...
	\end{solution}
	
	\item
	\begin{lightgraybox}
		${n \choose k} < {n \choose k + 1}$ for $k + 1 \leq \frac{n}{2}$
	\end{lightgraybox}
	\begin{solution}
		content...
	\end{solution}
	
	\item
	\begin{lightgraybox}
		${n \choose k} + {n \choose k + 1} = {n + 1 \choose k + 1}$ for $k < n$
	\end{lightgraybox}
	\begin{solution}
		content...
	\end{solution}
	
	\item
	\begin{lightgraybox}
		$n \choose k$ is an integer
	\end{lightgraybox}
	\begin{solution}
		content...
	\end{solution}
	
	\item
	\begin{lightgraybox}
		if $p$ is prime and $1 \leq k \leq p^n - 1$, then $p^n \choose k$ is divisible by $p$.
	\end{lightgraybox}
	\begin{solution}
		content...
	\end{solution}
\end{enumerate}

\subsubsection*{11 (The Freshman's Dream)}
\begin{graybox}
	Let $R$ be a commutative ring with identity of prime characteristic $p$. If $a, b \in R$, then $(a \pm b)^{p^n} = a^{p^n} \pm b^{p^n}$ for all integers $n \geq 0$.
\end{graybox}
\begin{solution}
	content...
\end{solution}

\subsubsection*{12}
\begin{graybox}
	An element of a ring is \textbf{nilpotent} if $a^n = 0$ for some $n$. Prove that in a commutative ring $a + b$ is nilpotent if $a$ and $b$ are. Show that this result may be false if $R$ is not commutative. 
\end{graybox}
\begin{solution}
	content...
\end{solution}

\subsubsection*{13}
\begin{graybox}
	In a ring $R$ the following conditions are equivalent.
	\begin{enumerate}[(a)]
		\item $R$ has no nonzero nilpotent elements
		\item If $a \in R$ and $a^2 = 0$, then $a = 0$.
	\end{enumerate}
\end{graybox}
\begin{solution}
	content...
\end{solution}

\subsubsection*{14}
\begin{graybox}
	Let $R$ be a commutative ring with identity and prime characteristic $p$. The map $R \to R$ given by $r \mapsto r^p$ is a homomorphism of rings called the Frobenius homomorphism.
\end{graybox}
\begin{solution}
	content...
\end{solution}

\subsubsection*{15}

\begin{enumerate}[(a)]
	\item 
	\begin{graybox}
		Give an example of a nonzero homomorphism $f : R \to S$ of rings with identity such that $f(1_R) \neq 1_S$
	\end{graybox}
	\begin{solution}
		content...
	\end{solution}
	
	\item
	\begin{graybox}
		If $f : R \to S$ is an epimorphism of rings with identity, then $f(1_R) = 1_S$.
	\end{graybox}
	\begin{solution}
		content...
	\end{solution}
	
	\item
	\begin{graybox}
		If $f : R \to S$ is a homomorphism of rings with identity and $u$ is a unit in $R$ such that $f(u)$ is a unit in $S$, then $f(1_R) = 1_S$ and $f(u^{-1}) = f(u)^{-1}$.
	\end{graybox}
	\begin{solution}
		content...
	\end{solution}
\end{enumerate}

\subsubsection*{16}
\begin{graybox}
	Let $f : R \to S$ be a homomorphism of rings such that $f(r) \neq 0$ for some nonzero $r \in R$. If $R$ has an identity and $S$ has no zero divisors, then $S$ is a ring with identity $f(1_R)$.
\end{graybox}
\begin{solution}
	content...
\end{solution}

\subsubsection*{17}
\begin{enumerate}[(a)]
	\item 
	\begin{graybox}
		If $R$ is a ring, then so is $R^{\text{op}}$, where $R^{\text{op}}$ is defined as follows. The underlying set of $R^{\text{op}}$ is precisely $R$ and addition in $R^{\text{op}}$ coincides with addition in $R$. Multiplication in $R^{\text{op}}$, denoted $\circ$, is defined by $a \circ b = ba$, where $ba$ is the product in $R$. $R^{\text{op}}$ is called the \textbf{opposite ring} of $R$.
	\end{graybox}
	\begin{solution}
		content...
	\end{solution}
	
	\item 
	\begin{graybox}
		$R$ has an identity if and only if $R^{\text{op}}$ does.
	\end{graybox}
	\begin{solution}
		content...
	\end{solution}
	
	\item 
	\begin{graybox}
		$R$ is a division ring if and only if $R^{\text{op}}$ is
	\end{graybox}
	\begin{solution}
		content...
	\end{solution}
	
	\item 
	\begin{graybox}
		$\left(R^{\text{op}}\right)^{\text{op}} = R$
	\end{graybox}
	\begin{solution}
		content...
	\end{solution}
	
	\item
	\begin{graybox}
		If $S$ is a ring, then $R \cong S$ if and only if $R^{\text{op}} \cong S^{\text{op}}$.
	\end{graybox}
\end{enumerate}

\subsubsection*{18}
\begin{graybox}
	Let $\QQ$ be the field of rational numbers and $R$ any ring. If $f,g : \QQ \to R$ are homomorphisms of rings such that $f\mid\ZZ = g\mid\ZZ$, then $f = g$.
\end{graybox}

\section{Ideals}
\subsection{Exercises}

\subsubsection*{1}
\begin{graybox}
	The set of all nilpotent elements in a commutative ring forms an ideal.
\end{graybox}
\begin{solution}
	content...
\end{solution}

\subsubsection*{2}
\begin{graybox}
	Let $I$ be an ideal in a commutative ring $R$ and let $\text{Rad } I = \{r \in R \mid r^n \in I \text{ for some } n\}$. Show that $\text{Rad } I$ is an ideal.
\end{graybox}
\begin{solution}
	content...
\end{solution}

\subsubsection*{3}
\begin{graybox}
	If $R$ is a ring and $a \in R$, then $J = \{r \in R \mid ra = 0\}$ is a left ideal and $K = \{r \in R \mid ar = 0\}$ is a right ideal in $R$.
\end{graybox}
\begin{solution}
	content...
\end{solution}

\subsubsection*{4}
\begin{graybox}
	If $I$ is a left ideal of $R$, then $A(I) = \{r \in R \mid rx = 0, \forall x \in I\}$ is an ideal in $R$.
\end{graybox}
\begin{solution}
	content...
\end{solution}

\subsubsection*{5}
\begin{graybox}
	If $I$ is an ideal in a ring $R$, let $[R:I]$ = $\{r \in R \mid xr \in I, \forall x \in R\}$. Prove that $[R:I]$ is an ideal of $R$ which contains $I$.
\end{graybox}
\begin{solution}
	content...
\end{solution}

\subsubsection*{6}
\begin{enumerate}[(a)]
	\item
	\begin{graybox}
		The center of the ring $S$ of all $2\times 2$ matrices over a field $F$ consists of all matrices of the form 
		$
		\begin{pmatrix}
			a & 0\\
			0 & a
		\end{pmatrix}
		$
	\end{graybox}
	\begin{solution}
		content...
	\end{solution}
	
	\item
	\begin{graybox}
		The center of $S$ is not an ideal in $S$.
	\end{graybox}
	\begin{solution}
		content...
	\end{solution}
	
	\item
	\begin{graybox}
		What is the center of the ring of all $n \times n$ matrices over a division ring?
	\end{graybox}
	\begin{solution}
		content...
	\end{solution}
\end{enumerate}

\subsubsection*{7}
\begin{enumerate}[(a)]
	\item 
	\begin{graybox}
		A ring $R$ with identity is a division ring if and only if $R$ has no proper left ideals.
	\end{graybox}
	\begin{solution}
		content...
	\end{solution}
	
	\item
	\begin{graybox}
		If $S$ is a ring (possibly without identity) with no proper left ideals, then either $S^2 = 0$ or $S$ is a division ring.
	\end{graybox}
	\begin{solution}
		content...
	\end{solution}
\end{enumerate}

\subsubsection*{8}
\begin{graybox}
	Let $R$ be a ring with identity and $S$ the ring of all $n \times n$ matrices over $R$. Then $J$ is an ideal of $S$ if and only if $J$ is the ring of all $n \times n$ matrices over $I$ for some ideal $I$ in $R$.
\end{graybox}
\begin{solution}
	content...
\end{solution}


\subsubsection*{9}
\begin{graybox}
	Let $S$ be the ring of all $n \times n$ matrices over a division ring $D$.
\end{graybox}
\begin{enumerate}[(a)]
	\item 
	\begin{lightgraybox}
		$S$ has no proper ideals (that is, 0 is a maximal ideal.)
	\end{lightgraybox}
	\begin{solution}
		content...
	\end{solution}
	
	\item
	\begin{lightgraybox}
		0 is a prime ideal which does not satisfy condition (1) of Theorem 2.15.
	\end{lightgraybox}
	\begin{solution}
		content...
	\end{solution}
\end{enumerate}

\subsubsection*{10}
\begin{enumerate}[(a)]
	\item 
	\begin{graybox}
		Show that $\ZZ$ is a principal ideal ring
	\end{graybox}
	\begin{solution}
		content...
	\end{solution}
	\item
	\begin{graybox}
		Every homomorphic image of a principal ideal ring is also a principal ideal ring.
	\end{graybox}
	\begin{solution}
		content...
	\end{solution}
	\item
	\begin{graybox}
		$\ZZ_m$ is a principal ideal ring for every $m > 0$.
	\end{graybox}
	\begin{solution}
		content...
	\end{solution}
\end{enumerate}

\subsubsection*{11}
\begin{graybox}
	If $N$ is the ideal of all nilpotent elements in a commutative ring $R$, then $R / N$ is a ring with no nonzero nilpotent elements.
\end{graybox}
\begin{solution}
	content...
\end{solution}

\subsubsection*{12}
\begin{graybox}
	Let $R$ be a ring without identity and with no zero divisors. Let $S$ be the ring whose additive group is $R \times \ZZ$ as in the proof of Theorem 1.10. Let $A = \{(r,n) \in S \mid rx + nx = 0, \forall x \in R\}$.
\end{graybox}
\begin{enumerate}[(a)]
	\item
	\begin{lightgraybox}
		$A$ is an ideal in $S$.
	\end{lightgraybox}
	\begin{solution}
		content...
	\end{solution}
	
	\item
	\begin{lightgraybox}
		$S / A$ has an identity and contains a subring isomorphic to $R$.
	\end{lightgraybox}
	\begin{solution}
		content...
	\end{solution}
	
	\item
	\begin{lightgraybox}
		$S / A$ has no zero divisors. 
	\end{lightgraybox}
	\begin{solution}
		content...
	\end{solution}
\end{enumerate}

\subsubsection*{13}
\begin{graybox}
	Let $f : R \to S$ be a homomorphism of rings, $I$ an ideal in $R$, and $J$ an ideal in $S$.
\end{graybox}
\begin{enumerate}[(a)]
	\item 
	\begin{lightgraybox}
		$f^{-1}(J)$ is an ideal in $R$ that contains $\ker f$.
	\end{lightgraybox}
	\begin{solution}
		content...
	\end{solution}
	
	\item
	\begin{lightgraybox}
		If $f$ is an epimorphism, then $f(I)$ is an ideal in $S$. If $f$ is not surjective, $f(I)$ need not be an ideal in $S$.
	\end{lightgraybox}
	\begin{solution}
		content...
	\end{solution}
\end{enumerate}

\subsubsection*{14}
\begin{graybox}
	If $P$ is an ideal in a not necessarily commutative ring $R$, then the following conditions are equivalent.
	\begin{enumerate}[(a)]
		\item $P$ is a prime ideal.
		\item If $r,s \in R$ are such that $rRs \subset P$, then $r \in P$ or $s in P$.
		\item If $(r)$ and $(s)$ are principal ideals of $R$ such that $(r)(s) \subset P$, then $r \in P$ or $s \in P$.
		\item If $U$ and $V$ are right ideals in $R$ such that $UV \subset P$, then $U \subset P$ or $V \subset P$.
		\item If $U$ and $V$ are left ideals in $R$ such that $UV \subset P$, then $U \subset P$ or $V \subset P$.
	\end{enumerate}

\end{graybox}
	\begin{solution}
	content...
\end{solution}

\subsubsection*{15}
\begin{graybox}
	The set consisting of zero and all zero divisors in a commutative ring with identity contains at least one prime ideal.
\end{graybox}
\begin{solution}
	content...
\end{solution}

\subsubsection*{16}
\begin{graybox}
	Let $R$ be a commutative ring with identity and suppose that the ideal $A$ of $R$ is contained in a finite union of prime ideals $P_1 \cup \dots \cup P_n$. Show that $A \subset P_i$ for some $i$.
\end{graybox}
\begin{solution}
	content...
\end{solution}

\subsubsection*{17}
\begin{graybox}
	Let $f : R \to S$ be an epimorphism of rings with kernel $K$.
\end{graybox}
\begin{enumerate}[(a)]
	\item 
	\begin{lightgraybox}
		If $P$ is a prime ideal in $R$ that contains $K$, then $f(P)$ is a prime ideal in $S$.
	\end{lightgraybox}
	\begin{solution}
		content...
	\end{solution}
	
	\item
	\begin{lightgraybox}
		If $Q$ is a prime ideal in $S$, then $f^{-1}(Q)$ is a prime ideal in $R$ that contains $K$.
	\end{lightgraybox}
	\begin{solution}
		content...
	\end{solution}
	
	\item
	\begin{lightgraybox}
		There is a one-to-one correspondence between the set of all prime ideals in $R$ that contain $K$ and the set of all prime ideals in $S$, given by $P \mapsto f(P)$. 
	\end{lightgraybox}
	\begin{solution}
		content...
	\end{solution}
	
	\item
	\begin{lightgraybox}
		If $I$ is an ideal in a ring $R$, then every ideal in $R / I$ is of the form $P / I$, where $P$ is a prime ideal in $R$ that contains $I$.
	\end{lightgraybox}
	\begin{solution}
		content...
	\end{solution}
\end{enumerate}

\subsubsection*{18}
\begin{graybox}
	An ideal $M \neq R$ in a commutative ring $R$ with identity is maximal if and only if for every $r \in R \setminus M$, there exists $x \in R$ such that $I_R - rx \in M$. 
\end{graybox}
\begin{solution}
	content...
\end{solution}

\subsubsection*{19}
\begin{graybox}
	The ring $E$ of even integers contains a maximal ideal $M$ such that $E / M$ is \textit{not} a field.
\end{graybox}
\begin{solution}
	content...
\end{solution}

\subsubsection*{20}
\begin{graybox}
	In the ring $\ZZ$ the following conditions on a nonzero ideal $I$ are equivalent
	\begin{enumerate}[(i)]
		\item $I$ is prime
		\item $I$ is maximal
		\item $I = (p)$ with $p$ prime. 
	\end{enumerate}
\end{graybox}
\begin{solution}
	content...
\end{solution}

\subsubsection*{21}
\begin{graybox}
	Determine all prime and maximal ideals in the ring $\ZZ_m$.
\end{graybox}
\begin{solution}
	content...
\end{solution}

\subsubsection*{22}
\begin{enumerate}[(a)]
	\item 
	\begin{graybox}
		If $R_1,\dots,R_n$ are rings with identity and $I$ is an ideal in $R_1 \times \dots \times R_n$, then $I = A_1 \times \dots \times A_m$, where each $A_i$ is an ideal in $R_i$.
	\end{graybox}
	\begin{solution}
		content...
	\end{solution}
	
	\item
	\begin{graybox}
		Show that the conclusion of (a) need not hold if the rings $R_i$ do not have identities.
	\end{graybox}
	\begin{solution}
		content...
	\end{solution}
\end{enumerate}

\subsubsection*{23}
\begin{graybox}
	An element $e$ in a ring $R$ is said to be \textbf{idempotent} if $e^2 = e$. An element of the center of the ring $R$ is said to be \textbf{central}. If $e$ is a central idempotent in a ring $R$ with identity, then 
\end{graybox}
\begin{enumerate}[(a)]
	\item 
	\begin{lightgraybox}
		$1_r - e$ is a central idempotent
	\end{lightgraybox}
	\begin{solution}
		content...
	\end{solution}
	
	\item
	\begin{lightgraybox}
		$eR$ and $(I_r - e)R$ are ideals in $R$ such that $R = eR \times (1_R - e)R$.
	\end{lightgraybox}
	\begin{solution}
		content...
	\end{solution}
\end{enumerate}

\subsubsection*{24}
\begin{graybox}
	Idempotent elements $e_1,\dots,e_n$ in a ring $R$ are said to be \textbf{orthogonal} if $e_ie_j = 0$ for $i \neq j$. If $R,R_1,\dots,R_n$ are rings with identity, then the following conditions are equivalent:
	\begin{enumerate}[(a)]
		\item $R \cong R_1 \times \dots \times R_n$
		\item $R$ contains a set of orthogonal central idempotents $\{e_1,\dots, e_n\}$ such that $e_1 + e_2 + \dots + e_n = 1_R$ and $e_iR \cong R_i$ for each $i$.
		\item $R$ is the internal direct product $R = A_1 \times \dots \times A_n$ where each $A_i$ is an ideal of $R$ such that $A_i \cong R_i$.
	\end{enumerate}
\end{graybox}
\begin{solution}
	content...
\end{solution}

\subsubsection*{25}
\begin{graybox}
	If $m \in \ZZ$ has a prime decomposition $m = p_1^{k_1}\dots p_t^{k_t}$ ($k_i > 0$; $p_i$ distinct primes) then there is an isomorphism of rings $\ZZ_m \cong Z_{p_1^{k_1}} \times \dots \times Z_{p_t^{k_t}}$
\end{graybox}
\begin{solution}
	content...
\end{solution}

\subsubsection*{26}
\begin{graybox}
	If $R = \ZZ, A_1 = (6)$ and $A_2 = (4)$, then the map $\theta : R/A_i \cap A_2 \to R/A_1 \times R/A_2$ of Corollary 2.27 is not surjective.
\end{graybox}
\begin{solution}
	content...
\end{solution}