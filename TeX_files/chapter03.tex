\chapter{Rings}


\section{Rings and Homomorphisms}
\subsection{Exercises}
\subsubsection{1}
\begin{enumerate}[(a)]
	\item
	\begin{graybox}
		Let $G$ be an (additive) abelian group. Define an operation of multiplication in $G$ by $ab = 0$ (for all $a, b\in G$). Then $G$ is a ring.
	\end{graybox}
	\begin{solution}
		By assumption, $(G, +)$ is an abelian group so condition (i) is satisfied. Condition (ii) is satisfied since
		\begin{align*}
			(ab)c &= 0c\\
			&= 0\\
			&= a0\\
			&= a(bc)
		\end{align*}
		for any $a,b,c \in G$. Finally,
		\begin{align*}
			a(b + c) &= 0\\
			&= 0 + 0\\
			&= ab + ac
		\end{align*}
		is true for every $a,b,c \in G$ and an analogous argument shows the right distributive law holds as well. Thus condition (iii) is satisfied and so by definition, $G$ is a ring. 
	\end{solution}
	\item
	\begin{graybox}
		Let $S$ be the set of all subsets of some fixed set $U$. For $A, B \in S$, define $A + B = (A \setminus B) \cup (B \setminus A)$ and $AB = A \cap B$. Then $S$ is a ring
	\end{graybox}
	\begin{solution}
		First, we show that $(S, +)$ is an abelian group. 
	\end{solution}
	\begin{lightgraybox}
		Is $S$ commutative?
	\end{lightgraybox}
	\begin{solution}
		content...
	\end{solution}
	\begin{lightgraybox}
		Does it have an identity?
	\end{lightgraybox}
	\begin{solution}
		content...
	\end{solution}
\end{enumerate}